\documentclass{article}
\usepackage[utf8]{inputenc}
\usepackage{amsmath,cancel}
\usepackage[spanish]{babel}
\title{Tarea \# 4, tarea grupal}
\author{David Esteban González González 2020425932\\Derek Umaña Quirós
2019208874\\Jefferson Arias Gutiérrez 2021131112\\María José Venegas Díaz 
202151065\\Brainer Chacón Orellana 2021039600}
\date{March 2022}
\begin{document}

\maketitle
Calcular las derivadas por definición de las funciones:\\
    Definición de una derivada:\\
\medium
\begin{equation*}
    \lim_{h\to 0} \frac{f(x+h)-f(x)}{h}
\end{equation*}
\\
$f(x) = mx+b$
\begin{equation*} \label{eq1}
\begin{split}
 \\\lim_{h\to 0} \frac{m(x+h)+b-(mx+b)}{h}\\
     &=\lim_{h\to 0}\frac{m(x+h)+\cancel{b}-mx\cancel{-b}}{h}\\
     \\
     &=\lim_{h\to 0}\frac{mx+mh-mx}{h}\\
     \\
     &=\lim_{h\to 0}\frac{\cancel{mx}+mh\cancel{-mx}}{h}\\
     \\
     &=\lim_{h\to 0}\frac{mh}{h}\\
      \\
     &=\lim_{h\to 0}\frac{m\cancel{h}}{\cancel{h}}\\
     \\
     &=m
\end{split}
\end{equation*}
\\
$g(x) = x^3$
\begin{equation*} \label{eq1}
\begin{split}
 \\\lim_{h\to 0} \frac{(x+h)^3-x^3}{h}\\
     &=\lim_{h\to 0}\frac{x^3+3x^2h+3xh^2+h^3-x^3}{h}\\
     \\
     &=\lim_{h\to 0}\frac{\cancel{x^3}+3x^2h+3xh^2+h^3\cancel{-x^3}}{h}\\
     \\
     &=\lim_{h\to 0}\frac{3x^2h+3xh^2+h^3}{h}\\
     \\
     &=\lim_{h\to 0}\frac{h(3x^2+3xh+h^2)}{h}\\
     \\
     &=\lim_{h\to 0}\frac{\cancel{h}(3x^2+3xh+h^2)}{\cancel{h}}\\
     \\
     &=\lim_{h\to 0}3x^2+3xh+h^2\\
     \\
     &=3x^2+3\cdot0x+0^2\\
     \\
     &=3x^2\\
\end{split}
\end{equation*}
\\
$g(x) = c$ constante
\begin{equation*} 
\begin{split}
 \\\lim_{h\to 0}\frac{c-c}{h}\\
     &=\lim_{h\to 0}\frac{\cancel{c}\cancel{-c}}{h}\\
     \\
     &=\lim_{h\to 0}\frac{0}{h}\\
     &=\cancelto{0}{\lim_{h\to 0}\frac{0}{h}}\\
     \\
     &=0
\end{split}
\end{equation*}

$g(x) = \sqrt{x+1}$ 
\begin{equation*} 
\begin{split}
 \\\lim_{h\to 0}\frac{\sqrt{x+h+1}-\sqrt{x+1}}{h}\\
     &=\lim_{h\to 0}\frac{\sqrt{x+h+1}-\sqrt{x+1}}{h}\cdot \frac{\sqrt{x+h+1}+\sqrt{x+1}}{\sqrt{x+h+1}+\sqrt{x+1}} \\
     \\
     &=\lim_{h\to 0}\frac{x+h+1-(x+1)}{h(\sqrt{x+h+1}+\sqrt{x+1})}\\
     \\
     &=\lim_{h\to 0}\frac{\cancel{x}+h\cancel{+1}\cancel{-x}\cancel{-1}}{h(\sqrt{x+h+1}+\sqrt{x+1})}\\
     \\
     &=\lim_{h\to 0}\frac{\cancel{h}}{\cancel{h}(\sqrt{x+h+1}+\sqrt{x+1})}\\
     \\
     &=\lim_{h\to 0}\frac{1}{\sqrt{x+h+1}+\sqrt{x+1}}\\
     \\
     &=\frac{1}{\sqrt{x+0+1}+\sqrt{x+1}}\\
     \\
     &=\frac{1}{\sqrt{x+1}+\sqrt{x+1}}\\
     \\
     &=\frac{1}{2\sqrt{x+1}}\\
\end{split}
\end{equation*}

$g(x) = \cos x$ 
\begin{equation*} 
\begin{split}
 \\\lim_{h\to 0}\frac{\cos(x+h)-\cos(x)}{h}\\
     &=\lim_{h\to 0}\frac{\cos(x+h)-\cos(x)}{h}\\
     \\
     &=\lim_{h\to 0}\frac{\cos(x)\cdot\cos(h)-\sin(x)\cdot \sin(h)-\cos(x)}{h}\\
     \\
     &=\lim_{h\to 0}\frac{[\cos(x)\cdot\cos(h)-\cos(x)]-\sin(x)\cdot \sin(h)}{h}\\
     \\
     &=\lim_{h\to 0}\frac{\cos(x)[\cos(h)-1]-\sin(x)\cdot \sin(h)}{h}\\
     \\
     &=\lim_{h\to 0}\frac{\cos(x)[\cos(h)-1]}{h}+\lim_{h\to 0}\frac{-\sin(x)\cdot \sin(h)}{h} \\\
     \\
     &=\cos(x)\lim_{h\to 0}(\frac{[\cos(h)-1]}{h})-\sin(x)\lim_{h\to 0} \frac{\sin(h)}{h} \\
     \\
     &=-\cos(x)\cancelto{0}{\lim_{h\to 0}(\frac{[1-\cos(h)]}{h})}-\sin(x)\cancelto{1}{\lim_{h\to 0} \frac{\sin(h)}{h}} \\
     \\
     &=-\cos(x)\cancelto{0}{\lim_{h\to 0}(\frac{[1-\cos(h)]}{h})}-\sin(x)\cancelto{1}{\lim_{h\to 0} \frac{\sin(h)}{h}} \\
     \\
     &=-\cos(x)\cdot0-\sin(x)\cdot 1 \\
     \\
     &=-\sin(x)\\
\end{split}
\end{equation*}
\newpage

Encuentre la recta tangente a la curva $y=x^3$ en el punto $(2,8)$

\begin{equation*}
    f(x)=x^3\\
\end{equation*}
\begin{equation*}
    f^\prime(x)=3x^2
\end{equation*}\\
Una vez tenemos la recta tangente, evaluamos en $f^\prime$, con esto obtendremos la pendiente de la recta tangente.
\begin{equation*}
    f^\prime(2)=3\cdot2^2\\
\end{equation*}
\begin{equation*}
    m=12
\end{equation*}
Ahora con la ecuación punto-pentdiente obtenemos la ecuación de la recta:
\begin{equation*}
    y-y_0=m(x-x_0)
\end{equation*}
sustituimos valores y luego despejamos y:
\begin{equation*}
    y-8=12(x-2)
\end{equation*}
la ecuación de la recta tangente a $x^3$ en el punto (2,8) es:
\begin{equation*}
    y=12x-16
\end{equation*}
\end{document}


