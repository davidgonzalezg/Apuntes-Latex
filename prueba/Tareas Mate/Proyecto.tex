%Preámbulo
\documentclass[stu, 12pt, letterpaper, donotrepeattitle, floatsintext, natbib]{apa7}
\usepackage[utf8]{inputenc}
\usepackage{comment}
\usepackage{marvosym}
\usepackage{graphicx}
\usepackage{float}
\usepackage{multicol}
\usepackage{blindtext}
\usepackage{multicol}
\usepackage{tikz}
\usepackage[normalem]{ulem}
\usepackage[spanish]{babel} 
\usepackage{siunitx}
\usepackage{tkz-euclide}

\selectlanguage{spanish}
\useunder{\uline}{\ul}{}
\newcommand{\myparagraph}[1]{\paragraph{#1}\mbox{}\\}

% Portada
\thispagestyle{empty}
\title{\Large Proyecto Cálculo Diferencial e Integral}
\author{David Esteban González González (2020425932) \\Derek Umaña Quirós
(2019208874) \\Jefferson Arias Gutiérrez (2021131112)\\María José Venegas Díaz 
(202151065)\\Brainer Chacón Orellana (2021039600)\\Raquel Peraza Garita (2021032478)} % (autores separados, consultar al docente)
% Manera oficial de colocar los autores:
%\author{Autor(a) I, Autor(a) II, Autor(a) III, Autor(a) X}
\affiliation{Instituto Tecnológico de Costa Rica}
\course{MA1102 Cálculo Diferencial e Integral\\Grupo 02\\Subgrupo 04}
\professor{Manuel Calderón Solano}
\duedate{Prime Semestre 2022}
\begin{document}
\maketitle
% Índices
\pagenumbering{roman}
    % Contenido
\renewcommand\contentsname{\largeÍndice}
\tableofcontents
\setcounter{tocdepth}{2}
\newpage
    % Fíguras
\renewcommand{\listfigurename}{\largeÍndice de fíguras}
\listoffigures
\newpage
    % Tablas
\renewcommand{\listtablename}{\largeÍndice de tablas}
\listoftables
\newpage

% Cuerpo
\pagenumbering{arabic}

% Referencias
\renewcommand\refname{\large\textbf{Referencias}}
\bibliography{mibibliografia}{\large Ejercicios del Proyecto}
    \subsection{Ejercicio 1. Optimización}
        ASADA MONTAÑA VERDE\\
        Este año la Asada Montaña Verde adquirió un terreno para construir un tanque de almacenamiento
        con el objetivo de mejorar la presión con que llega el agua a las casas más cercanas.
        Se necesitan almacenar 23 000 litros de agua potable, gastando la menor cantidad de material
        en el tanque. Diseña con tu equipo de trabajo la forma del tanque para la ASADA y,
        con derivadas, determina las medidas que minimicen el gasto.
        \subsubsection{Solución}
        \begin{multicols}{2}
                \begin{tikzpicture}
                        \tkzInit[xmin=0,xmax=7,ymax=8]
                        \tkzClip
                        %\tkzGrid
                        \tkzDefPoints{3.5/1/A, 3.5/5/B} ;
                        \draw[thick,blue] (A) ellipse (2 and 0.5);
                        \draw[thick,blue] (B) ellipse (2 and 0.5);
                        \draw[thick,blue] (1.5,1) -- (1.5,5);
                        \draw[thick,blue] (5.5,1) -- (5.5,5);  
                        \draw[dashed,white,thick] (5.5,1) arc (0:180:2 and 0.5);
                        \tkzLabelSegment[below](A,B){23 000 L}; 
                \end{tikzpicture}
                \caption{Figura 1}
                
        \end{multicols}

\newpage



\end{document}