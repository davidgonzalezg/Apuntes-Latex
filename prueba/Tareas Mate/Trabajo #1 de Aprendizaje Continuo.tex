%%%%%%%%%%%%%%%%%%%%%%%%%%%%% Define Article %%%%%%%%%%%%%%%%%%%%%%%%%%%%%%%%%%
\documentclass{article}
%%%%%%%%%%%%%%%%%%%%%%%%%%%%%%%%%%%%%%%%%%%%%%%%%%%%%%%%%%%%%%%%%%%%%%%%%%%%%%%

%%%%%%%%%%%%%%%%%%%%%%%%%%%%% Using Packages %%%%%%%%%%%%%%%%%%%%%%%%%%%%%%%%%%
\usepackage{geometry}
\usepackage{graphicx}
\usepackage{amssymb}
\usepackage{amsmath,cancel}
\usepackage{amsthm}
\usepackage{empheq}
\usepackage{mdframed}
\usepackage{booktabs}
\usepackage{lipsum}
\usepackage{graphicx}
\usepackage{color}
\usepackage{psfrag}
\usepackage{pgfplots}
\usepackage{bm}
%%%%%%%%%%%%%%%%%%%%%%%%%%%%%%%%%%%%%%%%%%%%%%%%%%%%%%%%%%%%%%%%%%%%%%%%%%%%%%%

% Other Settings

%%%%%%%%%%%%%%%%%%%%%%%%%% Page Setting %%%%%%%%%%%%%%%%%%%%%%%%%%%%%%%%%%%%%%%
\geometry{a4paper}

%%%%%%%%%%%%%%%%%%%%%%%%%% Define some useful colors %%%%%%%%%%%%%%%%%%%%%%%%%%
\definecolor{ocre}{RGB}{243,102,25}
\definecolor{mygray}{RGB}{243,243,244}
\definecolor{deepGreen}{RGB}{26,111,0}
\definecolor{shallowGreen}{RGB}{235,255,255}
\definecolor{deepBlue}{RGB}{61,124,222}
\definecolor{shallowBlue}{RGB}{235,249,255}
%%%%%%%%%%%%%%%%%%%%%%%%%%%%%%%%%%%%%%%%%%%%%%%%%%%%%%%%%%%%%%%%%%%%%%%%%%%%%%%

%%%%%%%%%%%%%%%%%%%%%%%%%% Define an orangebox command %%%%%%%%%%%%%%%%%%%%%%%%
\newcommand\orangebox[1]{\fcolorbox{ocre}{mygray}{\hspace{1em}#1\hspace{1em}}}
%%%%%%%%%%%%%%%%%%%%%%%%%%%%%%%%%%%%%%%%%%%%%%%%%%%%%%%%%%%%%%%%%%%%%%%%%%%%%%%

%%%%%%%%%%%%%%%%%%%%%%%%%%%% English Environments %%%%%%%%%%%%%%%%%%%%%%%%%%%%%
\newtheoremstyle{mytheoremstyle}{3pt}{3pt}{\normalfont}{0cm}{\rmfamily\bfseries}{}{1em}{{\color{black}\thmname{#1}~\thmnumber{#2}}\thmnote{\,--\,#3}}
\newtheoremstyle{myproblemstyle}{3pt}{3pt}{\normalfont}{0cm}{\rmfamily\bfseries}{}{1em}{{\color{black}\thmname{#1}~\thmnumber{#2}}\thmnote{\,--\,#3}}
\theoremstyle{mytheoremstyle}
\newmdtheoremenv[linewidth=1pt,backgroundcolor=shallowGreen,linecolor=deepGreen,leftmargin=0pt,innerleftmargin=20pt,innerrightmargin=20pt,]{theorem}{Theorem}[section]
\theoremstyle{mytheoremstyle}
\newmdtheoremenv[linewidth=1pt,backgroundcolor=shallowBlue,linecolor=deepBlue,leftmargin=0pt,innerleftmargin=20pt,innerrightmargin=20pt,]{definition}{Definition}[section]
\theoremstyle{myproblemstyle}
\newmdtheoremenv[linecolor=black,leftmargin=0pt,innerleftmargin=10pt,innerrightmargin=10pt,]{problem}{Problem}[section]
%%%%%%%%%%%%%%%%%%%%%%%%%%%%%%%%%%%%%%%%%%%%%%%%%%%%%%%%%%%%%%%%%%%%%%%%%%%%%%%

%%%%%%%%%%%%%%%%%%%%%%%%%%%%%%% Plotting Settings %%%%%%%%%%%%%%%%%%%%%%%%%%%%%
\usepgfplotslibrary{colorbrewer}
\pgfplotsset{width=8cm,compat=1.9}
%%%%%%%%%%%%%%%%%%%%%%%%%%%%%%%%%%%%%%%%%%%%%%%%%%%%%%%%%%%%%%%%%%%%%%%%%%%%%%%

%%%%%%%%%%%%%%%%%%%%%%%%%%%%%%% Aprendizaje continuo David González %%%%%%%%%%%%%%%%%%%%%%%%%%%%%%%%
\title{Trabajo \#1 de Aprendizaje Continuo }
\author{David González, 2020425932}
%%%%%%%%%%%%%%%%%%%%%%%%%%%%%%%%%%%%%%%%%%%%%%%%%%%%%%%%%%%%%%%%%%%%%%%%%%%%%%%

\begin{document}
  
\maketitle
Link del video:
\begin{equation*} \label{eq1}
    \begin{split}
    \\L=\lim_{x\to - \infty } \frac{-x}{\sqrt{4x^2+6}-9x}\\
        &=\lim_{x\to - \infty } \frac{-x}{\sqrt{x^2(4+\frac{6}{x^2})}-9x}\\\\
        &=\lim_{x\to - \infty } \frac{-x}{-x\sqrt{(4+\frac{6}{x^2})}-9x}\\\\
        &=\lim_{x\to - \infty } \frac{\cancel{-x}}{\cancel{-x}(\sqrt{(4+\frac{6}{x^2})}+9)}\\\\
        &=\lim_{x\to - \infty } \frac{1}{\sqrt{4+\frac{6}{x^2}}+9}\\\\
    \end{split}
\end{equation*}


    1.	¿Cuales considera que fueron sus habilidades destrezas durante el desarrollo de la prueba que le permitieron obtener esa nota? Mencione y explique minimo 2.\\
    La práctica constante, el problema usa muchas cosas de mate general, como factor común 
    y la simplificación, otra habilidad es el entendimiento de las acciones que se hacen en otros límites hechos en clase
    \\2.	¿Cómo ha logrado desarrollar dichas habilidades? Explique su respuesta.\\
    Práctica en verano, una introducción a la materia en clases previas
    \\3.	¿Considera qué sus conocimientos previos relacionados con los temas evaluados influyeron 
    en la nota obtenida? Explique su respuesta.\\
    Pues si el conocimiento de mate general ayuda para resolver esto ejercicios
    \\4.	¿Tiene hábitos de estudio particulares? 
    Si su respuesta es firmativa, mencione cuáles.
    Me cuesta, pero intento estudiar todas las semanas los que pide el programa del curso, en las tardes me siento a 
    ver si hago algunos ejecicios\\
    \\5.	¿Le gustan las matemáticas? ¿Ha tenido buenas experiencias con la matemática? \\
    Me gusta lo que se es capaz de hacer con ellas, por ejemplo vi que había una fórmula que le permitia a 
    uno como sacar todas las ondas de las que estaba compuesta una onda como abstracta, mi experiencia con las matemáticas
    ha sido regular, tampoco soy bueno con ellas, pero cuando me salen las cosas me pone feliz y también me gusta
    explicarle lo poquito que sé a los demás.
    \\6.	Mencione aspectos no considerados en los puntos anteriores que influyeron sobre la nota obtenida. 
    Por ejemplo, asuntos personales, actitud hacia la matemática, conocimientos previos, entre otros.\\
    Los conocimientos previos de verano y tutorías, a veces es difícil estudiar con la familia en la casa pero
    se intenta.

\end{document}