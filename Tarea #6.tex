%%%%%%%%%%%%%%%%%%%%%%%%%%%%% Define Article %%%%%%%%%%%%%%%%%%%%%%%%%%%%%%%%%%
\documentclass{article}
%%%%%%%%%%%%%%%%%%%%%%%%%%%%%%%%%%%%%%%%%%%%%%%%%%%%%%%%%%%%%%%%%%%%%%%%%%%%%%%

%%%%%%%%%%%%%%%%%%%%%%%%%%%%% Using Packages %%%%%%%%%%%%%%%%%%%%%%%%%%%%%%%%%%
\usepackage{geometry}
\usepackage{graphicx}
\usepackage{amssymb}
\usepackage{amsmath,cancel}
\usepackage{amsthm}
\usepackage{empheq}
\usepackage{mdframed}
\usepackage{booktabs}
\usepackage{lipsum}
\usepackage{graphicx}
\usepackage{color}
\usepackage{psfrag}
\usepackage{pgfplots}
\usepackage{bm}
%%%%%%%%%%%%%%%%%%%%%%%%%%%%%%%%%%%%%%%%%%%%%%%%%%%%%%%%%%%%%%%%%%%%%%%%%%%%%%%

% Other Settings

%%%%%%%%%%%%%%%%%%%%%%%%%% Page Setting %%%%%%%%%%%%%%%%%%%%%%%%%%%%%%%%%%%%%%%
\geometry{a4paper}

%%%%%%%%%%%%%%%%%%%%%%%%%% Define some useful colors %%%%%%%%%%%%%%%%%%%%%%%%%%
\definecolor{ocre}{RGB}{243,102,25}
\definecolor{mygray}{RGB}{243,243,244}
\definecolor{deepGreen}{RGB}{26,111,0}
\definecolor{shallowGreen}{RGB}{235,255,255}
\definecolor{deepBlue}{RGB}{61,124,222}
\definecolor{shallowBlue}{RGB}{235,249,255}
%%%%%%%%%%%%%%%%%%%%%%%%%%%%%%%%%%%%%%%%%%%%%%%%%%%%%%%%%%%%%%%%%%%%%%%%%%%%%%%

%%%%%%%%%%%%%%%%%%%%%%%%%% Define an orangebox command %%%%%%%%%%%%%%%%%%%%%%%%
\newcommand\orangebox[1]{\fcolorbox{ocre}{mygray}{\hspace{1em}#1\hspace{1em}}}
%%%%%%%%%%%%%%%%%%%%%%%%%%%%%%%%%%%%%%%%%%%%%%%%%%%%%%%%%%%%%%%%%%%%%%%%%%%%%%%

%%%%%%%%%%%%%%%%%%%%%%%%%%%% English Environments %%%%%%%%%%%%%%%%%%%%%%%%%%%%%
\newtheoremstyle{mytheoremstyle}{3pt}{3pt}{\normalfont}{0cm}{\rmfamily\bfseries}{}{1em}{{\color{black}\thmname{#1}~\thmnumber{#2}}\thmnote{\,--\,#3}}
\newtheoremstyle{myproblemstyle}{3pt}{3pt}{\normalfont}{0cm}{\rmfamily\bfseries}{}{1em}{{\color{black}\thmname{#1}~\thmnumber{#2}}\thmnote{\,--\,#3}}
\theoremstyle{mytheoremstyle}
\newmdtheoremenv[linewidth=1pt,backgroundcolor=shallowGreen,linecolor=deepGreen,leftmargin=0pt,innerleftmargin=20pt,innerrightmargin=20pt,]{theorem}{Theorem}[section]
\theoremstyle{mytheoremstyle}
\newmdtheoremenv[linewidth=1pt,backgroundcolor=shallowBlue,linecolor=deepBlue,leftmargin=0pt,innerleftmargin=20pt,innerrightmargin=20pt,]{definition}{Definition}[section]
\theoremstyle{myproblemstyle}
\newmdtheoremenv[linecolor=black,leftmargin=0pt,innerleftmargin=10pt,innerrightmargin=10pt,]{problem}{Problem}[section]
%%%%%%%%%%%%%%%%%%%%%%%%%%%%%%%%%%%%%%%%%%%%%%%%%%%%%%%%%%%%%%%%%%%%%%%%%%%%%%%

%%%%%%%%%%%%%%%%%%%%%%%%%%%%%%% Plotting Settings %%%%%%%%%%%%%%%%%%%%%%%%%%%%%
\usepgfplotslibrary{colorbrewer}
\pgfplotsset{width=8cm,compat=1.9}
%%%%%%%%%%%%%%%%%%%%%%%%%%%%%%%%%%%%%%%%%%%%%%%%%%%%%%%%%%%%%%%%%%%%%%%%%%%%%%%

%%%%%%%%%%%%%%%%%%%%%%%%%%%%%%% Title & Author %%%%%%%%%%%%%%%%%%%%%%%%%%%%%%%%
\title{Tarea \#6, tarea grupal}
\author{David Esteban González González 2020425932\\Derek Umaña Quirós
2019208874\\Jefferson Arias Gutiérrez 2021131112\\María José Venegas Díaz 
202151065\\Brainer Chacón Orellana 2021039600\\Raquel Peraza Garita 2021032478\\}
%%%%%%%%%%%%%%%%%%%%%%%%%%%%%%%%%%%%%%%%%%%%%%%%%%%%%%%%%%%%%%%%%%%%%%%%%%%%%%%

\begin{document}
    \maketitle
    \large
    Usando la diferencial\\
    Aproximar:\\
    \begin{equation*} 
        \begin{split}
         \\\sqrt{83}\\
             &f(x)=\sqrt x, f^\prime(x)= \frac{1}{2 \sqrt x} \\
             \\
             &a=81, \varDelta x=2\\
             \\
             &f(81 + 2)=f(81)+f^\prime(81)\cdot 2\\
             \\
             &\sqrt{83} = \sqrt{81} + \frac{1}{2\sqrt{81}}\cdot 2 \\
             \\
             &=9+\frac{1}{18}\cdot 2\\
             \\
             &=9+\frac{1}{9}\\
             \\
             &=\frac{81}{9}+\frac{1}{9}\\
             \\
             &=\frac{82}{9}\approx 9,11...\\
        \end{split}
    \end{equation*}


    \begin{equation*} 
        \begin{split}
         \\\log_{2}33\\
             &f(x)=\log_{2}x, f^\prime(x)=\frac{1}{x \ln2} \\
             \\
             &a=32, \varDelta x=1\\
             \\
             &f(32 + 1)=f(32)+f^\prime(32)\cdot 1\\
             \\
             &\log_{2}33 = \log_{2}32 + \frac{1}{32\ln2}\\
             \\
             &=5+\frac{1}{32\ln2}\approx 5,04...\\
             \\
        \end{split}
    \end{equation*}
    \\
    \begin{equation*} 
        \begin{split}
         \\\sqrt[3]{7}\\
             &f(x)=\sqrt[3]x, f^\prime(x)= \frac{1}{3 \sqrt[3]{x^2}} \\
             \\
             &a=8, \varDelta x=1\\
             \\
             &f(8 -1)=f(8)+f^\prime(8)\cdot 1\\
             \\
             &\sqrt[3]{8-1} = \sqrt[3]{8} + \frac{1}{3\sqrt[3]{8^2}}\cdot 1 \\
             \\
             &= 2 + \frac{1}{3\sqrt[3]{64}}\\
             \\
             &=2-\frac{1}{12}\\
             \\
             &=\frac{24}{12}-\frac{1}{12}\\
             \\
             &=\frac{23}{12}\approx 1,9...\\
        \end{split}
    \end{equation*}
    \\
    \begin{equation*} 
        \begin{split}
         \\\sin(46^\circ), 46^\circ \longrightarrow \frac{23\pi }{90}rad\\
             &\sin(\frac{23\pi }{90})\\
             \\
             &f(x)=\sin(x), f^\prime(x)=\cos(x) \\
             \\
             &a=\frac{\pi}{4}, \varDelta x=\frac{\pi}{180}\\
             \\
             &f(\frac{\pi}{4}+\frac{\pi}{180})=f(\frac{\pi}{4})+f^\prime(\frac{\pi}{4})\cdot \frac{\pi}{180}\\
             \\
             &= \sin(\frac{\pi}{4}) + \cos(\frac{\pi}{4})\cdot \frac{\pi}{180} \\
             \\
             &= \frac{\sqrt{2}}{2}+\frac{\sqrt{2}}{2}\cdot \frac{\pi}{180}\\
             \\
             &=\frac{\sqrt{2}}{2}+\frac{\pi\sqrt{2}}{360}\\
             \\
             &=\frac{180\sqrt{2}}{360}+\frac{\pi\sqrt{2}}{360}\\
             \\
             &=\frac{180\sqrt{2}+\pi \sqrt{2}}{360}\\
             \\
             &=\frac{(180+\pi) \sqrt{2}}{360}\approx 0,72...\\
        \end{split}
    \end{equation*}
\end{document}

