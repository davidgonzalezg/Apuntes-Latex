\documentclass[a4paper, 12pt, addpoints]{exam}
\usepackage{IfbaProvas}
\printanswers% para imprimir soluções
%\noprintanswers % Não imprimir soluções
%==============================================================
\tipoAvaliacao{Avaliação \textcolor{red}{\textbf{Far From Home}}}
%==============================================================
\input{InfGerais}
%==============================================================
%Comando para colocar ou NÃO imagem de fundo, não precisa comentar
\ifdef{\fundo}{\backgroundsetup{contents={\fundo},        placement=top,scale=1,angle=0,opacity=0.2,position={7.95cm,2.55cm}}}{\newpage\backgroundsetup{contents={}}}
%==============================================================

%COMEÇO DO DOCUMENTO
\begin{document}

\info\vspace{-1.5 cm} %Imprime as informações do cabeçalho programada no pacote 
%TABELA DE PONTUAÇÃO E PESOS. Se não quiser é só excluir
\begin{center}
	\gradetable[h]
\end{center}

\begin{questions}%comece a escrever as questões
\question[2] Dada a função $f(x)=\sqrt{x+1}-\sqrt{x}$, responda às questões, justificando:
\begin{parts}
\part A função é crescente ou decrescente?\\
\part Calcule $\lim_{x\to\infty} f(x)$\\
 \end{parts}
\question[2] Verifique se as hipóteses do teste da integral são satisfeitas para as séries abaixo. Em caso positivo, aplique-o e mostre se as séries são convergentes ou divergentes.
\begin{multicols}{4}
\begin{enumerate}[a)]
\item $\sum_{n=1}^\infty \frac{1}{n^2+1}$ 
\item $\sum_{n=1}^\infty \frac{\ln n}{n}$\label{testar}
\item $\sum_{n=1}^\infty \frac{n^2}{n+1}$ 
\item $\sum_{n=1}^\infty \frac{1}{n}$
\end{enumerate}
\end{multicols}
\question[\half] Mostre pelo método da comparação direta se a série do item (\ref{testar}, da questão anterior, converge ou diverge.

\question[1\half] Utilizando a expansão em série de potências das funções envolvidas, calcule:
\begin{multicols}{2}
\begin{enumerate}[a)]
\item $\lim _{x \rightarrow 0} \frac{\operatorname{sen} x}{x}$
\item $\int x\sin(x^3)\;dx$
\end{enumerate}
\end{multicols}
\question[2] Resolva os problemas de valor inicial
\begin{multicols}{2}
\begin{enumerate}[a)]
    \item $\left\{\begin{array}{l} y'=-yx\\\\ y(1) = 1\end{array}\right.$
    \item $\left\{\begin{array}{l} y''-e^x+\cos x=0\\\\y(0)=2\\ y'(0)=3\end{array}\right.$
\end{enumerate}
\end{multicols}
\question[2] As questões acima servem apenas como exemplos

\end{questions}
\end{document}